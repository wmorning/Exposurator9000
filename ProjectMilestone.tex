%%%%%%%%%%%%%%%%%%%%%%%%%%%%%%%%%%%%%%%%%%%%%%%%%%%%%%%%%%%%%%%%%%%%%%
% Machine Learning Milestone report
%%%%%%%%%%%%%%%%%%%%%%%%%%%%%%%%%%%%%%%%%%%%%%%%%%%%%%%%%%%%%%%%%%%%%%

\documentclass[iop,apj]{emulateapj}
\usepackage{amsmath,amssymb,amstext}

% PACKAGES AND MARGINS:

\usepackage[breaklinks,colorlinks,citecolor=blue,linkcolor=blue]{hyperref}
\usepackage[all]{hypcap} %Links go to figures; breaks on deluxetables (use \capstartfalse \capstarttrue to fix it)
\usepackage{float}
\addtolength\voffset{0.8cm}

% FORMATTING ETC:

\newcommand{\ds}{\displaystyle}
\newcommand{\be}{\begin{equation}}
\newcommand{\ee}{\end{equation}}
\newcommand{\bee}{\begin{eqnarray}}
\newcommand{\eee}{\end{eqnarray}}

% MATH MACROS:

\def\pr{{\rm Pr}}
\def\pars{\mathbf{x}}
\def\parsi{x_i}
\def\data{\mathbf{d}}
\def\datai{d_i}
\def\datap{\mathbf{d}^{\rm p}(\pars)}
\def\datapi{d^{\rm p}_{i}(\pars)}

% TEXT MACROS:

\def\etal{\hbox{et al.} }
\def\ie{{\it i.e.\ }}
\def\eg{{\it e.g.\ }}

% COMMENTING ETC:

\usepackage[usenames]{color}
\newcommand{\comment}[1]{\textcolor{blue}{\bf #1}}
\newcommand{\question}[1]{\textcolor{blue}{\bf #1}}
\newcommand{\todo}[2]{\textcolor{red}{\bf To do (#1): #2}}

%%%%%%%%%%%%%%%%%%%%%%%%%%%%%%%%%%%%%%%%%%%%%%%%%%%%%%%%%%%%%%%%%%%%%%

\begin{document}
\title{Project Milestone:  Searching for Artifacts in images from the Dark Energy Survey}
\author{Joseph W. DeRose}
\author{Warren R. Morningstar}
\affil{Kavli Institute for Particle Astrophysics and Cosmology
  \& Physics Department, Stanford University, Stanford, CA 94305, USA}




\begin{abstract}
In this paper, we present an implementation of a Machine Learning algorithm
to be used in identifying and classifying image artifacts in observations taken by the Dark 
Energy Survey (DES).  Specifically, we treat our individual pixels as features, and run a 
classifier using a Support Vector Machine.  We find that our first implementation identifies
image artifacts with an accuracy of {\color{red} Some Percent}.  We discuss future improvements
to our modeling, and sources of systematic errors in our analysis, and how these errors may be 
overcome.

\end{abstract}

\section{Introduction} \label{sec:intro}

In modern cosmology, much attention has been turned toward using large photometric sky 
surveys to provide strong constraints on structure formation and the composition of our 
universe.  An important component of modern sky surveys is the management of the 
large volumes of data produced by the survey instruments.  Typical data generation of 
current sky surveys is several GB of data per night, and future sky surveys will produce
TB of data in the same time frame.  Because of the vast quantities of data produced, 
the majority of analysis is performed in an automated (or at least semi-automated) manner.

While this provides significant advantages in performance, an obvious disadvantage is that
it is prone to overlooking subtle features in the data, which may cause systematic errors
in measurements of important quantities.  In particular, since the goal of many photometric
surveys is to provide precise measurements of the brightness of astronomical objects, any
spurious brightness variations within a detector has the potential to be interpreted as a real
signal, and thus has the capability of interfering with the accuracy of measurements.

Brightness variations in a detector can be caused in many different ways, and thus have a 
diverse variety of appearances.  A few notable 
examples that have been observed by ongoing sky surveys include airplanes and satellites
passing across the field of view, cosmic rays striking the detector, and improper subtraction of 
the background noise.  Some of these are detected and masked away by the survey 
data reduction pipelines, but a large number are missed (these are usually referred to as
artifacts).  Reliable identification of artifacts is an essential component of preparing data for
analysis.  However, currently much of the identification of artifacts is carried out by members 
of survey collaborations, a process that uses a significant number of human hours and that 
could be better spent devoted to more scientifically interesting problems.  Therefore an 
automated means of identifying and classifying image artifacts is desirable.  

The Dark Energy Survey (DES) is a sky survey being conducted across many institutions.
Survey is being carried out using the 4m Blanco Telescope located at the Cerro Tololo Inter-American
Observatory in Chile.  The survey saw first light in 2012, and is currently in its 3rd year.  
In this paper, we implement an algorithm for identification of image artifacts found in observations
performed by DES.  We describe our data set in Section~\ref{sec:data}.  In Section~\ref{sec:methods}, we
describe our choice of features and the implementation of our model.  In Section~\ref{sec:results}, we 
present the results of preliminary classification runs.  In Section~\ref{sec:discussion}, we discuss these results, 
including ways in which our modeling could be improved.

\section{Data} \label{sec:data}

{\color{red}  Describe our data set.  Include any undersampling of images we used, and how we intend to
classify images.  Maybe a sample image or two, although I dunno if we're allowed to do that on account of the data is private.}

\section{Methods} \label{sec:methods}

{\color{red} SVM implementation.  Feature selection.  Any struggles we had along the way to show the volume of work we've done}

\section{Results} \label{sec:results}

{\color{red} How good did the SVM do?  A few preliminary quantitative results would be useful (e.g. test error for the SVM).  Maybe a plot or two showing some key feature that distinguishes artifacts from non-artifacts.  Not sure if there will even be such a thing.  Maybe cool to include a plot of a population of successfully identified artifacts}

\section{Discussion} \label{sec:discussion} 

{\color{red} Probably most of the milestone report to go here.  Discuss limitations of SVM for the problem we're 
trying to accomplish.  Say what we're gonna do moving forward.  If no good results to show, explain what is going wrong currently.}






\end{document}